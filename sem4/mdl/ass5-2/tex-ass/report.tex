\documentclass[10pt,letterpaper]{article}

\usepackage{amsmath}
\usepackage{alltt}
\usepackage{amssymb}
\usepackage{chemfig}
\usepackage{mathtools}
\usepackage{listings}
\usepackage{wrapfig}
\usepackage{color}
\usepackage{float}
\usepackage{caption}
\usepackage{subcaption}
\usepackage{paralist}
\usepackage{tcolorbox}
\usepackage{braket}
\usepackage[framemethod=TikZ]{mdframed}
\usepackage[english]{babel}
\usepackage[utf8x]{inputenc}
\usepackage{esint}
\usepackage{hyperref}
\hypersetup{
    colorlinks=true,
    linkcolor=blue,
    filecolor=magenta,      
    urlcolor=cyan,
}

\title{MDL Assignment 5}
\date{Part 2}
\author{Kalp Shah : 2018113003}

\setlength{\columnsep}{20pt}

\begin{document}
\pagenumbering{arabic}
\maketitle

\section*{Question 1}
If you know the target is in (1,1) cell and your observation is o6 , what will be
the initial belief state? Please submit the optimal policy file named
(RollNumber).policy for the POMDP taking into account the initial belief
state you obtained.
\section*{Answer 1}
Answer is in the policy file

\section*{Question 2}
If you are in (0,1) and you know the target is in your one neighborhood and
is not making a call what is your initial belief state?
\section*{Answer 2}
Given : \\
\hspace*{23pt} Call : Off\\
\hspace*{23pt} Agent Coordinates : (0,1)\\
\hspace*{23pt} Possible Target Coordinates : (0,1), (0,0), (1,1) or (2,0)\\\\
The only quantitiy that changes is the Target Coordinates.\\\\
$\therefore$ For the system with Agent at (0,1) and Call : Off, the 
probabilities are as follows :
\begin{itemize}
    \item Target (0,1)\\ \hspace*{23pt} P(E) = 0.25
    \item Target (0,0)\\ \hspace*{23pt} P(E) = 0.25
    \item Target (1,1)\\ \hspace*{23pt} P(E) = 0.25
    \item Target (2,0)\\ \hspace*{23pt} P(E) = 0.25
\end{itemize}
Where E $\equiv$ Given Event

\section*{Question 3}
What is the expected utility for initial belief states in questions 1 and 2?
\section*{Answer 3}
The expected utility is :
\begin{itemize}
    \item Question 1\\ \hspace*{23pt} Utility = 1.045
    \item Question 2\\ \hspace*{23pt} Utility = 3.267
\end{itemize}

\section*{Question 4}
If your agent is in (0,1) with probability 0.6 and in (2,1) with probability 0.4
and the target is in the 4 corner cells with equal probability, which
observation are you most likely to observe? Explain.
\section*{Answer 4}
Given : \\
\hspace*{23pt} Agent : P(S = (0,1)) = 0.6 , P(S = (2,1)) = 0.4\\
\hspace*{23pt} Target : (0,0), (0,2), (2,0) or (2,2)
\\\\
All probabilities are initially 0, hence :
\begin{equation*}
    P(O_i) = 0, \forall i \in [0,6]
\end{equation*}

\noindent$\therefore$ For Agent at (0,1), the probabilities are :
\begin{itemize}
    \item Target (0,0)\\ \hspace*{23pt} P($O_4$) += 0.6 * 0.25\\ \hspace*{52pt} += 0.15
    \item Target (0,2)\\ \hspace*{23pt} P($O_2$) += 0.6 * 0.25\\ \hspace*{52pt} += 0.15
    \item Target (2,0)\\ \hspace*{23pt} P($O_6$) += 0.6 * 0.25\\ \hspace*{52pt} += 0.15
    \item Target (2,2)\\ \hspace*{23pt} P($O_6$) += 0.6 * 0.25\\ \hspace*{52pt} += 0.15
\end{itemize}

\noindent$\therefore$ For Agent at (2,1), the probabilities are :
\begin{itemize}
    \item Target (0,0)\\ \hspace*{23pt} P($O_6$) += 0.4 * 0.25\\ \hspace*{52pt} += 0.10
    \item Target (0,2)\\ \hspace*{23pt} P($O_6$) += 0.4 * 0.25\\ \hspace*{52pt} += 0.10
    \item Target (2,0)\\ \hspace*{23pt} P($O_4$) += 0.4 * 0.25\\ \hspace*{52pt} += 0.10
    \item Target (2,2)\\ \hspace*{23pt} P($O_2$) += 0.4 * 0.25\\ \hspace*{52pt} += 0.10
\end{itemize}

The final probabilities, thus are :\\
    \hspace*{23pt} P($O_2$) = 0.25\\
    \hspace*{23pt} P($O_4$) = 0.25\\
    \hspace*{23pt} P($O_6$) = 0.50\\

\noindent Hence, $max(P(O_i)) \equiv P(O_6) = 0.50$, which is believable since 
for each agent coordinate, the probabilites remain same, but the observation 
corrosponding to $O_6$ is twice. (Due to there being 2 target states giving 
$O_6$) 

\section*{Question 5}
How many policy trees are obtained in this case, explain?
\section*{Answer 5}
No of policy trees is given by this formulae :\\
\begin{equation*}
    N \equiv \sum_{i = 0}^{T-1} |O|^i = \frac{|O|^T - 1}{|O| - 1}
\end{equation*}
\begin{equation*}
    N_{Tree} = A^N
\end{equation*}

\noindent In the computations done above, 
the values required in the equations are :\\
\hspace*{23pt} T = 1000\\
\hspace*{23pt} $|O|$ = 6\\
\hspace*{23pt} A = 5\\

\begin{align*}
    \implies N & = \frac{6^{1000} - 1}{6 - 1}\\
    & \equiv \frac{6^{1000}}{5}\\
    & \rightarrow \infty
\end{align*}

\begin{align*}
    \therefore N_{Tree} & = A^\infty\\
    & \rightarrow \infty
\end{align*}

\noindent There are problems with this computation, as not all actions are available 
for each observation and thus the actual value would be a little less than 
the computed one.
\end{document}
