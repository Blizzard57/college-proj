\chapter{Sequential RNG}
In this section, the most common sequential RNGs are 
introduced. This is required as RNGs that work in parallel 
are extensions of SRNGs, and understanding a simpler system 
before introducing a N processor generality wil be the 
logical and simpler way to understand PRNGs.

\section{Linear Congruential Generators}
Linear Congruential Generators, abbreviated as LCGs are 
defined as follows:

\begin{definition}
    fFor a sequence of numbers $X_1,X_2, \dots, X_n $ with 
    $n^{th}$ number being $X_n$, 
    the sequence is :\\
    \begin{equation*}
        X_n = aX_{n-1} + c \mod m
    \end{equation*}

    \noindent Where $a,c,m$ are parameters.
\end{definition}

\noindent This is one of the most common RNG, which has been 
used for a long time. One advantage that this construct has 
is that it is very quick in generation when $m$ is chosen 
to be a power of 2.
\\\\
The disadvantage of the system are that it produces highly 
correlated lower order bits for intervals that are power of 
2. It also has the issue that the modulus cannot be greater 
than the precision fo the machine.
\\\\
One modification that is commonly used for the algorithm is 
generalization of LCG, known as Multiple Recursive Generators 
(MRGs), which are defined as:

\begin{definition}
    fFor a sequence of numbers $X_1,X_2, \dots, X_n $ with 
    $n^{th}$ number being $X_n$, 
    the sequence is :\\
    \begin{equation*}
        X_n = \sum_{i=1}^{n-k} a_iX_{i} + c \mod m
    \end{equation*}

    \noindent Where $a_i,c,m,k$ are parameters.
\end{definition}

\noindent This is a recursion based some of the previous 
values rather than just the last one. The benefit of this 
construct is that the randomness of the system increases, 
lesser the $k$ is, on the downside that it takes a longer 
time to compute.

\subsection{Code}
\lstinputlisting[basicstyle=Large,style=py]{code/lcg.py}

\section{Lagged-Fibonacci Generators}
Lagged-Fibonacci Generators, abbreviated as LFGs are 
defined as follows:

\begin{definition}
    fFor a sequence of numbers $X_1,X_2, \dots, X_n $ with 
    $n^{th}$ number being $X_n$, 
    the sequence is :\\
    \begin{equation*}
        X_n = X_{n-p} \odot X_{n-q} \mod m
    \end{equation*}

    \noindent Where $p,q,m$ are parameters and $\odot$ is 
    any binary operation (Add, Multiply ,XOR).
\end{definition}

\noindent From the formulation itself it can be seen that 
the usage of the construct requires $p (p > q)$ initial values 
to be already present.
\\\\
The binary operation can be selected by the user and it has 
been proven that multiply operation provides the most amount 
of randomness followed by the addition operation and lastly 
followed by XOR operation.
\\\\
The major benifit of the construct is that it is extremely 
quick in computing the value. Its period can also be made 
arbitrarily wide by just changing the value of p.

\subsection{Code}
\lstinputlisting[basicstyle=Large,style=py]{code/lfg.py}

\section{Combined Generators}
This is a system which just combines two psuedo 
random sequences, which in turn gives a random sequence 
which would have a better quality of randomness. 
Hence, the formulation is as follows :

\begin{definition}
    fFor two sequences $X_n$ and $Y_n$, making a combined 
    sequence $S_n$ as follows:\\
    \begin{equation*}
        S_n = X_{n} + Y_{n} \mod m
    \end{equation*}

    \noindent Where $m$ is the parameter.
\end{definition}

\section{Conclusion}
It is visible that psuedo random techniques used are not 
that complicated, but rather tricky. The combination of 
the use of recursion and modulo provides us with psuedo 
random numbers. 
\\\\
The improvement in quality can be performed 
by performing binary operations on these sequences, which 
if having different parameters would have different patterns 
thus obscuring the pattern even more, making the operated 
sequence even more random.